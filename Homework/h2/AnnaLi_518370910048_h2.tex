\documentclass[a4paper,12pt]{journal}
\usepackage[dvipsnames, svgnames, x11names]{xcolor} 
\usepackage{amsmath}
\usepackage{amssymb}
\usepackage[margin=2.5cm]{geometry}
\usepackage{graphics}
\usepackage{ulem}
\usepackage{setspace}
\usepackage{listings}
\usepackage{algorithm}  
\usepackage{algpseudocode}  
\usepackage{amsmath}  
\usepackage{xcolor}
\usepackage[greek,english]{babel}
\usepackage{chemformula}
\usepackage{wrapfig}
\usepackage{multirow}
\usepackage{booktabs}
\usepackage{fancyhdr}
\usepackage{pgfplots}
\usepackage{tikz}
\pagestyle{fancy}
\rmfamily
\fancyhf{}
\fancyfoot[R]{\thepage}
\fancyhead[R]{VE475 HW2}
\title{VE475 Homework2}
\author{Anna Li \\Student ID: 518370910048}
\date{\today}
\lstset{
	columns=fixed,     
	numbers=left,                                        % 在左侧显示行号
	numberstyle=\tiny\color{gray},                       % 设定行号格式
	frame=none,                                          % 不显示背景边框
	backgroundcolor=\color[RGB]{245,245,244},            % 设定背景颜色
	keywordstyle=\color[RGB]{40,40,255},                 % 设定关键字颜色
	numberstyle=\footnotesize\color{darkgray},           
	commentstyle=\it\color[RGB]{0,96,96},                % 设置代码注释的格式
	stringstyle=\ttfamily\slshape\color[RGB]{128,0,0},   % 设置字符串格式
	showstringspaces=false,                              % 不显示字符串中的空格                                        % 设置语言
}
\begin{document}
	\maketitle
	\section*{Ex. 1}
	\subsection*{1}
	Since we need to find the inverse of 17 modulo 101, we need to find an integer x, which is $17x\equiv 1\mod 101$. Therefore, according to extended Euclidean algorithm, 
	\begin{equation}
	\begin{array}{r c l}
		101 &=& 17*5+16\\
		17&=&16+1\\
	\end{array}
	\end{equation}
Therefore,
$$1=17-16=17-(101-17*5)=17*6-101$$
Therefore, the inverse of 17 modulo 101 is 6
	\subsection*{2}
	since $12x\equiv 28\mod 236$,
	$$12x+236y=28\Rightarrow3x+59y=7$$
	First, we found that x=22 is the only integer solution for this equation when x<59.Therefore:
	$$x=59n+22,n\in\mathbb{R}$$
	\subsection*{3}
	since plaintext=m modulo 31, we could know that $x\in[0,30]$. $c\in[0,30]$
	\begin{center}
	\begin{tabular}{c c c c c c}
		m&c&m&c&m&c\\
		0&0&1&1&2&4\\
		3&17&4&16&5&5\\
		6&6&7&28&8&2\\
		9&10&10&20&11&13\\
		12&24&13&22&14&19\\
		15&23&16&8&17&12\\
		18&9&19&7&20&18\\
		21&11&22&21&23&29\\
		24&3&25&25&26&26\\
		27&15&28&14&29&27\\
		30&30&31&0&32&1\\
	\end{tabular}
	\end{center}
Therefore, we can decrypt this message.
\subsection*{4}
Since $\sqrt{4883}=69.9\quad\sqrt{4369}=66.09$\\
Therefore, we should consider 1,2,3,5,7,9,11,13,17,19,23,29,31,37,41,43,47,53,59,61,67\\
Therefore:
$$4883=19*257\quad 4369=17*257$$
\subsection*{5}
After calculation, we found that only when p=2, $det(A\mod p)=0$. Therefore, when p is not equal to 2, this equation is not invertible.
\subsection*{6}
since p is a prime, and $ab\equiv 0 \mod p$, we know that $ab=np,n\in\mathbb{Z}$. Then we prove: the statement "none of a and b is congruent to 0 mod p" is wrong.\\
If none of a and b is congruent to 0 mod p, which means that $a\not =xp,x\in\mathbb{Z}$ and $b\not =yp,y\in\mathbb{Z}$.Since p is a prime number, $ab\not = mp,m\in\mathbb{Z}$, which is not true. Therefore, there are at least one of a and b is congruent to 0 mod p.\\
\subsection*{7}
since
$$2\equiv 2\mod 5\quad2^2\equiv 4\mod 5\quad2^3\equiv 3\mod 5\quad2^4\equiv 1\mod 5\quad2^5\equiv 4\mod 5$$
Therefore, $$2^{2017}\equiv 1\mod 5$$
$$2\equiv 2\mod 13\quad2^2\equiv 4\mod 13\quad2^3\equiv 8\mod 13\quad2^4\equiv 3\mod 13\quad2^5\equiv 12\mod 13$$
$$2^6\equiv 9\mod 13\quad2^7\equiv 5\mod 13\quad2^8\equiv 10\mod 13\quad2^9\equiv 7\mod 13\quad2^{10}\equiv 1\mod 13$$
Therefore, $$2^{2017}\equiv 7\mod 5$$
	s
\end{document}
