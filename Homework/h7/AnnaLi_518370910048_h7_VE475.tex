\documentclass[a4paper,12pt]{journal}
\usepackage[dvipsnames, svgnames, x11names]{xcolor} 
\usepackage{amsmath}
\usepackage{amssymb}
\usepackage[margin=2.5cm]{geometry}
\usepackage{graphics}
\usepackage{ulem}
\usepackage{setspace}
\usepackage{listings}
\usepackage{algorithm}  
\usepackage{algpseudocode}  
\usepackage{amsmath}  
\usepackage{xcolor}
\usepackage[greek,english]{babel}
\usepackage{chemformula}
\usepackage{wrapfig}
\usepackage{multirow}
\usepackage{booktabs}
\usepackage{fancyhdr}
\usepackage{pgfplots}
\usepackage{enumerate} % List
\usepackage{tikz}
\pagestyle{fancy}
\rmfamily
\fancyhf{}
\fancyfoot[R]{\thepage}
\fancyhead[R]{VE475 HW7}
\title{VE475 Homework7}
\author{Anna Li \\Student ID: 518370910048}
\date{\today}
\lstset{
	columns=fixed,     
	numbers=left,                                        % 在左侧显示行号
	numberstyle=\tiny\color{gray},                       % 设定行号格式
	frame=none,                                          % 不显示背景边框
	backgroundcolor=\color[RGB]{245,245,244},            % 设定背景颜色
	keywordstyle=\color[RGB]{40,40,255},                 % 设定关键字颜色
	numberstyle=\footnotesize\color{darkgray},           
	commentstyle=\it\color[RGB]{0,96,96},                % 设置代码注释的格式
	stringstyle=\ttfamily\slshape\color[RGB]{128,0,0},   % 设置字符串格式
	showstringspaces=false,                              % 不显示字符串中的空格                                        % 设置语言
}
\begin{document}
	\maketitle
\section*{Ex. 1 — Cramer-Shoup cryptosystem}
\subsection*{1.}
The encryption algorithm: \\
After Bob sends Alice plaintext m, Bob maps m to an element of G. Then he chooses a random number r which is $0\leq r\leq p-1$. Last Bob calculates $u_1=g_1^r$, $u_2=g_2^r$, $e=h^rm$, $a=H(u_1,u_2,e)$ and $v=e^kd^{aR}$. Then ciphertext is c$=(u_1,u_2,e,v)$.\\
The decryption algorithm: \\
If $u_1^{x_1+ay_1}u_2^{x_2yy_2a}=v$, the plaintext m would be $\frac{e}{u_1^{z}}$. The decryption will fail in other way.\\
The generation of the key algorithm: \\
After a cyclic group G with order p is generated by Alice, she finds two key generators g1 and g2 for G. Alice chooses $x_1,x_2,y_1,y_2,z$which in $\{1,2,....,p-1\}$ and computes ciphertext$g_1^{x_1}$,$g_2^{x_2}$, d=$g_1^{y_1}g_2^{y_2}$ , h=$g_1^{z}$. H is a one-way trapdoor function be generated. Lastly, private keys are $x_1,x_2,y_1,y_2,z$ and c,d,h,G,p,$g_1,g_2$ are public keys.\\ 
\subsection*{2.}
CCA could be used. Because the attackers could attack by conducting different ciphertexts. However, the decryption algorithm doesn't allow all ciphertexts because of one way collision-resistant hash function.
\subsection*{3.}
Similarity\\
 Calculation is in symmetric group is both hard to decrypt due to discrete logarithm problems.\\
Difference:\\
Another protecting layer is added for Cramer algorithm (trpdoor one-way hash function). Moreover, it can restrict the input of cipher. Therefore, Cramer-Shoup system is safer. \\
\section*{Ex. 2 — Simple questions}
\subsection*{1.}
 If p is a prime and p $\not | \alpha$ ,which means gcd(p,$\alpha$)=1 and we can get $\alpha^{p-1} \equiv 1\mod p$ by Euler's theorem. Since the hash function isn't second pre-image resistant, since x is known, we can find $x'=x+(p-1)$ and $h(x)=h(x')$, which is not collision resistant too. Therefore it's not a good cryptographic hash function.\\
\subsection*{2.}
$$2^{30}=0x40000000$$
$$\Rightarrow floor(2^{30}*\sqrt{2})= 5A827999$$
$$\Rightarrow floor(2^{30}*\sqrt{3})= 6ED9EBA1$$
$$\Rightarrow floor(2^{30}*\sqrt{5})= 8F1BBCDC$$
$$\Rightarrow floor(2^{30}*\sqrt{10})=  CA62C1D6$$
Therefore, the result is the same as $K_0||......||K19,K20||......||K39,K40||......||K59 andK60||......||K79.$\\
\section*{Ex. 3 — Birthday paradox}
\subsection*{1.}
$$g(x) = ln(1-x)+x+x2$$
$$\Rightarrow g'(x)=-\frac{1}{1-x}+1+2x$$
When x=0 or 0.5$\Rightarrow$g'(x)=0\\
$$\Rightarrow g''(x)=-\frac{1}{x-1}^{2}+2$$
Therefore, g(0)=1 is local min, g(0.5)=-2 is local max.\\
When $x\in[0,0.5]$, $g(x)\in[g(0),g(0.5)]>=0$\\
Similarly, we set h(x) = ln(1 - x) + x\\
$$\Rightarrow h(x)\in[g(0.5),g(0)]<=0$$
Therefore, $-x-x^2 <= ln(1-x) <= -x$\\
\subsection*{2.}
$j\in[1,r-1]$ and $r\leq \frac{n}{2}$ $\Rightarrow\frac{j}{n}\in\left[0,\frac{1}{2}\right]$
$$\Rightarrow -\frac{j}{n}-\left(\frac{j}{n}\right)^2 \leq \ln\left(1-\frac{j}{n}\right) \leq -\frac{j}{n}$$
$$\Rightarrow\sum_{j=1}^{r-1} \left[ -\frac{j}{n}-\left(\frac{j}{n}\right)^2 \right] \leq \sum_{j=1}^{r-1} \ln\left(1-\frac{j}{n}\right) \leq \sum_{j=1}^{r-1} -\frac{j}{n}$$
$$\Rightarrow-\frac{(r-1)r}{2n}-\frac{(r-1)r(2r-1)}{6n^2} \leq \sum_{j=1}^{r-1} \ln\left(1-\frac{j}{n}\right) \leq -\frac{(r-1)r}{2n}$$
\\
When $r>1$, \\
$\frac{(r-1)r(2r-1)}{6n^2}=\frac{r^3-\frac{3}{2}r^2+r}{3n^2}<\frac{r^3}{3n^2}$\\
Therefore,\\
$-\frac{(r-1)r}{2n}-\frac{r^3}{3n^2} \leq \sum_{j=1}^{r-1} \ln\left(1-\frac{j}{n}\right) \leq -\frac{(r-1)r}{2n}$\\
\subsection*{3.}
$$\exp\left(-\frac{(r-1)r}{2n}-\frac{r^3}{3n^2}\right) \leq \prod_{j=1}^{r-1} \left(1-\frac{j}{n}\right) \leq \exp\left(-\frac{(r-1)r}{2n}\right)$$

Supposing $\lambda=\dfrac{r^2}{2n}$, $$c_1=\sqrt{\dfrac{\lambda}{2}}-\dfrac{(2\lambda)^{3/2}}{3}$$ $$c_2=\sqrt{\dfrac{\lambda}{2}}$$
$$\Rightarrow-\lambda+\frac{c_1}{\sqrt{n}}=-\frac{r^2}{2n}+\frac{r}{2n}-\frac{r^3}{n^2}=-\frac{(r-1)r}{2n}-\frac{r^3}{3n^2}$$
$$\Rightarrow-\lambda+\frac{c_2}{\sqrt{n}}=-\frac{r^2}{2n}+\frac{r}{2n}=-\frac{(r-1)r}{2n}$$

Therefore,\\
$$e^{-\lambda}e^{c_1/\sqrt{n}} \leq \prod_{j=1}^{r-1} \left(1-\frac{j}{n}\right) \leq e^{-\lambda}e^{c_2/\sqrt{n}}$$
\subsection*{4.}
When n is large and $\lambda =\frac{r^2}{2n} <\frac{n}{8}$, $r<\frac{n}{2}$\\
Since $\lambda$ is constant, $c_1$ and $c_2$ are all constants.\\
$$\Rightarrow\lim n->\inf e^{\frac{c_1}{\sqrt{n}}}=1$$
$$\Rightarrow\lim n->\inf e^{\frac{c_2}{\sqrt{n}}}=1$$
Therefore, we can get: $\sum_{j=1}^{j=r-1}(1-\frac{j}{n})=e^{-\lambda}$ \\
\section*{Ex. 4 — Birthday attack}
\subsection*{1.}
0.546
\subsection*{2.}
0.039
\subsection*{3.}
It's easy to find a collision in a hash function, and it's  hard to find a collision of a specific massage. This means that Alice could overcome the problem by changing a bit in the massage but Eve cannot find a collision of new massage easily.\\
\section*{Ex. 5 — Faster multiple modular exponentiation}

 \end{document}
